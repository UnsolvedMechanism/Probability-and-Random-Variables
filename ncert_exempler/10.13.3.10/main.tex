\let\negmedspace\undefined
\let\negthickspace\undefined
\documentclass[journal,12pt,twocolumn]{IEEEtran}
\usepackage{cite}
\usepackage{amsmath,amssymb,amsfonts,amsthm}
\usepackage{algorithmic}
\usepackage{graphicx}
\usepackage{textcomp}
\usepackage{xcolor}
\usepackage{txfonts}
\usepackage{listings}
\usepackage{enumitem}
\usepackage{mathtools}
\usepackage{gensymb}
\usepackage[breaklinks=true]{hyperref}
\usepackage{tkz-euclide} % loads  TikZ and tkz-base
\usepackage{listings}
\usepackage{gvv}
%
%\usepackage{setspace}
%\usepackage{gensymb}
%\doublespacing
%\singlespacing

%\usepackage{graphicx}
%\usepackage{amssymb}
%\usepackage{relsize}
%\usepackage[cmex10]{amsmath}
%\usepackage{amsthm}
%\interdisplaylinepenalty=2500
%\savesymbol{iint}
%\usepackage{txfonts}
%\restoresymbol{TXF}{iint}
%\usepackage{wasysym}
%\usepackage{amsthm}
%\usepackage{iithtlc}
%\usepackage{mathrsfs}
%\usepackage{txfonts}
%\usepackage{stfloats}
%\usepackage{bm}
%\usepackage{cite}
%\usepackage{cases}
%\usepackage{subfig}
%\usepackage{xtab}
%\usepackage{longtable}
%\usepackage{multirow}
%\usepackage{algorithm}
%\usepackage{algpseudocode}
%\usepackage{enumitem}
%\usepackage{mathtools}
%\usepackage{tikz}
%\usepackage{circuitikz}
%\usepackage{verbatim}
%\usepackage{tfrupee}
%\usepackage{stmaryrd}
%\usetkzobj{all}
%    \usepackage{color}                                            %%
%    \usepackage{array}                                            %%
%    \usepackage{longtable}                                        %%
%    \usepackage{calc}                                             %%
%    \usepackage{multirow}                                         %%
%    \usepackage{hhline}                                           %%
%    \usepackage{ifthen}                                           %%
  %optionally (for landscape tables embedded in another document): %%
%    \usepackage{lscape}     
%\usepackage{multicol}
%\usepackage{chngcntr}
%\usepackage{enumerate}

%\usepackage{wasysym}
%\documentclass[conference]{IEEEtran}
%\IEEEoverridecommandlockouts
% The preceding line is only needed to identify funding in the first footnote. If that is unneeded, please comment it out.

\newtheorem{theorem}{Theorem}[section]
\newtheorem{problem}{Problem}
\newtheorem{proposition}{Proposition}[section]
\newtheorem{lemma}{Lemma}[section]
\newtheorem{corollary}[theorem]{Corollary}
\newtheorem{example}{Example}[section]
\newtheorem{definition}[problem]{Definition}
%\newtheorem{thm}{Theorem}[section] 
%\newtheorem{defn}[thm]{Definition}
%\newtheorem{algorithm}{Algorithm}[section]
%\newtheorem{cor}{Corollary}
\newcommand{\BEQA}{\begin{eqnarray}}
\newcommand{\EEQA}{\end{eqnarray}}
\newcommand{\define}{\stackrel{\triangle}{=}}
\theoremstyle{remark}
\newtheorem{rem}{Remark}

\begin{document}

\bibliographystyle{IEEEtran}


\vspace{3cm}

\title{
	Question: 10.13.3.10
}
\author{ Yash Patil - EE22BTECH11058
}	

\maketitle

\newpage

\renewcommand{\thefigure}{\theenumi}
\renewcommand{\thetable}{\theenumi}
Question: All the jacks, queens and kings are removed from a deck of 52 playing cards. The remaining cards are well shuffled and then one card is drawn at random. Giving ace a value 1 similar value for other cards, find the probability that the card has a value 
\begin{enumerate}
	\item 7
	\item greater than 7
	\item less than 7
\end{enumerate}
\solution
Number of cards left after removing all jacks, queens and kings(=N) 
\begin{align}
	&= 52 - 4*3\\
	&= 40
\end{align}
\begin{enumerate}
	\item Number of cards with value equal to 7($n_1$) = 4\\
		$\therefore$ Probability to find a card with value equal to 7 ($P_1$) 
		\begin{align}
			&= \frac{n_1}{N}\\
			&= \frac{4}{40}\\
			&=0.1
		\end{align}
	\item Number of cards with value greater than 7($n_2$) 
		\begin{align} = 4*3 = 12 \end{align}
		$\therefore$ Probability to find a card with value greater than 7 ($P_2$) 
		\begin{align}
			&= \frac{n_2}{N}\\
			&= \frac{12}{40}\\
			&=0.3
		\end{align}
	\item Number of cards with value less than 7($n_3$)
		\begin{align} = 4*6 = 24 \end{align}
		$\therefore$ Probability to find a card with value lesser than 7 ($P_3$) 
		\begin{align}
			&= \frac{n_3}{N}\\
			&= \frac{24}{40}\\
			&=0.6
		\end{align}
\end{enumerate}
\end{document}
