\let\negmedspace\undefined
\let\negthickspace\undefined
\documentclass[journal,12pt,twocolumn]{IEEEtran}
\usepackage{cite}
\usepackage{amsmath,amssymb,amsfonts,amsthm}
\usepackage{algorithmic}
\usepackage{graphicx}
\usepackage{textcomp}
\usepackage{xcolor}
\usepackage{txfonts}
\usepackage{listings}
\usepackage{enumitem}
\usepackage{mathtools}
\usepackage{gensymb}
\usepackage[breaklinks=true]{hyperref}
\usepackage{tkz-euclide} % loads  TikZ and tkz-base
\usepackage{listings}
\usepackage{float}
\newtheorem{theorem}{Theorem}[section]
\newtheorem{problem}{Problem}
\newtheorem{proposition}{Proposition}[section]
\newtheorem{lemma}{Lemma}[section]
\newtheorem{corollary}[theorem]{Corollary}
\newtheorem{example}{Example}[section]
\newtheorem{definition}[problem]{Definition}
\newcommand{\BEQA}{\begin{eqnarray}}
\newcommand{\EEQA}{\end{eqnarray}}
\newcommand{\define}{\stackrel{\triangle}{=}}
\theoremstyle{remark}
\newtheorem{rem}{Remark}
\parindent 0px

\begin{document}
%
\providecommand{\pr}[1]{\ensuremath{\Pr\left(#1\right)}}
\providecommand{\prt}[2]{\ensuremath{p_{#1}^{\left(#2\right)} }}        % own macro for this question
\providecommand{\qfunc}[1]{\ensuremath{Q\left(#1\right)}}
\providecommand{\sbrak}[1]{\ensuremath{{}\left[#1\right]}}
\providecommand{\lsbrak}[1]{\ensuremath{{}\left[#1\right.}}
\providecommand{\rsbrak}[1]{\ensuremath{{}\left.#1\right]}}
\providecommand{\brak}[1]{\ensuremath{\left(#1\right)}}
\providecommand{\lbrak}[1]{\ensuremath{\left(#1\right.}}
\providecommand{\rbrak}[1]{\ensuremath{\left.#1\right)}}
\providecommand{\cbrak}[1]{\ensuremath{\left\{#1\right\}}}
\providecommand{\lcbrak}[1]{\ensuremath{\left\{#1\right.}}
\providecommand{\rcbrak}[1]{\ensuremath{\left.#1\right\}}}
\newcommand{\sgn}{\mathop{\mathrm{sgn}}}
\providecommand{\abs}[1]{\left\vert#1\right\vert}
\providecommand{\res}[1]{\Res\displaylimits_{#1}} 
\providecommand{\norm}[1]{\left\lVert#1\right\rVert}
%\providecommand{\norm}[1]{\lVert#1\rVert}
\providecommand{\mtx}[1]{\mathbf{#1}}
\providecommand{\mean}[1]{E\left[ #1 \right]}
\providecommand{\cond}[2]{#1\middle|#2}
\providecommand{\fourier}{\overset{\mathcal{F}}{ \rightleftharpoons}}
\newenvironment{amatrix}[1]{%
  \left(\begin{array}{@{}*{#1}{c}|c@{}}
}{%
  \end{array}\right)
}
\newcommand{\solution}{\noindent \textbf{Solution: }}
\newcommand{\cosec}{\,\text{cosec}\,}
\providecommand{\dec}[2]{\ensuremath{\overset{#1}{\underset{#2}{\gtrless}}}}
\newcommand{\myvec}[1]{\ensuremath{\begin{pmatrix}#1\end{pmatrix}}}
\newcommand{\mydet}[1]{\ensuremath{\begin{vmatrix}#1\end{vmatrix}}}
\newcommand{\myaugvec}[2]{\ensuremath{\begin{amatrix}{#1}#2\end{amatrix}}}
\providecommand{\rank}{\text{rank}}
\providecommand{\pr}[1]{\ensuremath{\Pr\left(#1\right)}}
\providecommand{\qfunc}[1]{\ensuremath{Q\left(#1\right)}}
	\newcommand*{\permcomb}[4][0mu]{{{}^{#3}\mkern#1#2_{#4}}}
\newcommand*{\perm}[1][-3mu]{\permcomb[#1]{P}}
\newcommand*{\comb}[1][-1mu]{\permcomb[#1]{C}}
\providecommand{\qfunc}[1]{\ensuremath{Q\left(#1\right)}}

\providecommand{\gauss}[2]{\mathcal{N}\ensuremath{\left(#1,#2\right)}}
\providecommand{\diff}[2]{\ensuremath{\frac{d{#1}}{d{#2}}}}
\providecommand{\myceil}[1]{\left \lceil #1 \right \rceil }
\newcommand\figref{Fig.~\ref}
\newcommand\tabref{Table~\ref}
\newcommand{\sinc}{\,\text{sinc}\,}
\newcommand{\rect}{\,\text{rect}\,}
\let\StandardTheFigure\thefigure
\let\vec\mathbf


\bibliographystyle{IEEEtran}


\vspace{3cm}

\title{
%	\logo{
Q-10.13.3.10
%	}
}
\author{Yash Patil - EE22BTECH11058}

\maketitle
\textbf{Question:} Let $X_1$, $X_2$, ..., $X_{10}$ be a random sample of size 10 from a $N_3(\mu,\Sigma)$ distribution, where $\mu$ and a non-singular $\Sigma$ are unknown parameters. If
\begin{align}
	\overline{X_1} &= \frac{1}{5}\sum_{i=1}^5X_i\\
	\overline{X_2} &= \frac{1}{5}\sum_{i=6}^{10}X_i\\
	S_1 &= \frac{1}{4}\sum_{i=1}^5(X_i-\overline{X_1})(X_i-\overline{X_1})^\top\\
	S_2 &= \frac{1}{4}\sum_{i=6}^{10}(X_i-\overline{X_2})(X_i-\overline{X_2})^\top
\end{align}
Then which one of the following statements is not true?
\begin{enumerate}
	\item $\frac{5}{6}(\overline{X_1}-\mu)^{\top}S_1^-1(\overline{X_1}-\mu)$ follows a $F$-distribution with 3 and 2 degrees of freedom
	\item $\frac{6}{5(\overline{X_1}-\mu)^{\top}S_1^-1(\overline{X_1}-\mu)}$ follows a $F$-distribution with 3 and 2 degrees of freedom
	\item $4(S_1 + S_2)$ follows a Wishart distribution of order 3 with 8 degrees of freedom
	\item $5(S_1 + S_2)$ follows a Wishart distribution of order 3 with 10 degrees of freedom
\end{enumerate}
\solution\\
Covariance matrix, $\Sigma$ is defined as
\begin{align}
%	\Sigma &= E[(X-E[X])(Y-E[Y])]\\
%	&\text{as in this case X = Y}\\
	\Sigma &= E[(X-E[X])(X-E[X])^\top]\\
	&= \frac{1}{N-1}\sum_{i=1}^N(X_i-\overline{X})(X_i-\overline{X})^\top\\
	&\text{and,}\\
	S_1 &= \frac{1}{4}\sum_{i=1}^5(X_i-\overline{X_1})(X_i-\overline{X_1})^\top\\
	S_2 &= \frac{1}{4}\sum_{i=6}^{10}(X_i-\overline{X_2})(X_i-\overline{X_2})^\top
\end{align}
$\therefore$ $S_1$ and $S_2$ represent covariance matrix for their respective samples\\
\begin{enumerate}
	\item \textbf{For option 1 and 2:}\\
Given,
\begin{align}
	&X_i \sim N_3(\mu,\Sigma)\\
	\implies &\overline{X} \sim N_3(\mu, \Sigma/5)\\
	\therefore &\overline{X_1} \sim N_3(\mu,S_1/5)
\end{align}
converting to chi-squared distribution,
\begin{align}
	\implies &\frac{(\overline{X_1} - \mu)^2}{S_1/5} \sim \chi^2_3
\end{align}
As $\overline{X_1}-\mu$ represents a trivariate distribution, 
\begin{align}
	(\overline{X_1}-\mu)^2 = (\overline{X_1}-\mu)^{\top}(\overline{X_1}-\mu),
\end{align}
and bringing $S_1$ in numerator,
\begin{align}
	\implies 5(\overline{X_1}-\mu)^{\top}S_1^{-1}(\overline{X_1}-\mu) \sim \chi^2_3
\end{align}
%Converting $\chi^2$ distribution into F-distribution by dividing it with the degree of freedom,
%\begin{align}
%	\implies \frac{5}{3}(\overline{X_1}-\mu)^{\top}\Sigma^{-1}(\overline{X_1}-\mu) \sim \chi^2_3\\
%\end{align}
F distribution is defined as 
\begin{align}
	\frac{X/d_1}{Y/d_2} &\sim F(d_1,d_2) &\text{ where}\\
	X &\sim \chi^2_{d_1} \text{ and } Y \sim \chi^2_{d_2}
\end{align}
are independant chi-square distributions\\
Let X represent be a $\chi^2_3$ variable, then
\begin{align}
	\therefore \frac{X/3}{Y/d} &\sim F(3,d) \ \  \text{and}\\
	 \frac{Y/d}{X/3} &\sim F(d,3) \ \ \ \forall\ \ \  d \in \mathbb{N}
\end{align}
Hence for $d$ = 2,
\begin{align}
	\frac{5}{3}(\overline{X_1}-\mu)^{\top}\Sigma^{-1}(\overline{X_1}-\mu) &\sim F(3,2)\\
	\therefore \frac{5}{6}(\overline{X_1}-\mu)^{\top}\Sigma^{-1}(\overline{X_1}-\mu) &\sim F(3,2), \text{and similarly}\\
	\frac{6}{5(\overline{X_1}-\mu)^{\top}\Sigma^{-1}(\overline{X_1}-\mu)} &\sim F(2,3)
\end{align}
Hence, \textbf{option 1 and 2 are true}
\item \textbf{For option 3 and 4}\\
By defination, Wishart distribution is given by:
\begin{align}
	x_i &\sim N_p(\mu,\Sigma) \ \ \ \forall \ \ \ 1 \leq i \leq n\\
	M &= \sum_{i=1}^n x_ix_i^\top \sim W_p(\Sigma,n)
\end{align}
where p denotes order and n denotes the degree of freedom
\begin{align}
	X_i &\sim N_3(\mu,\Sigma) \ \ \forall \ \ 1\leq i\leq 10\\
	\implies Y_i &= X_i - \overline{X_1} \sim N_3(0,\Sigma) \ \ \forall\ \  1\leq i\leq 5, \text{and}\\
	Y_i &= X_i - \overline{X_2} \sim N_3(0,\Sigma) \ \ \forall\ \  6\leq i\leq 10
\end{align}
\begin{align}
	\therefore M_1 &= S_1 = \sum_{i=1}^{5}Y_iY_i^\top \sim W_{3}(\Sigma,4), \text{and}\\
	M_2 &= S_2 = \sum_{i=6}^{10}Y_iY_i^\top \sim W_{3}(\Sigma,4)\\
	\therefore M_1+M_2 &\sim W_3(\Sigma,4+4)\\
	\implies S_1+S_2 &\sim W_3(\Sigma,8)
\end{align}
		$\therefore \lambda(S_1+S_2)$ follow Wishart distribution with order 3 and 8 degrees of freedom where $\lambda \in \mathbb{R}$\\
Hence, \textbf{option 3 is True and option 4 is False}

\end{enumerate}
\end{document}
