%\let\negmedspace\undefined
%\let\negthickspace\undefined
\documentclass[journal,12pt,onecolumn]{IEEEtran}
%       \def\inputGnumericTable{}                                 %%
\usepackage{cite}
\usepackage{amsmath,amssymb,amsfonts,amsthm}
\usepackage{algorithmic}
\usepackage{graphicx}
\usepackage{textcomp}
\usepackage{xcolor}
\usepackage{txfonts}
\usepackage{listings}
\usepackage{enumitem}
\usepackage{mathtools}
\usepackage{gensymb}
\usepackage[breaklinks=true]{hyperref}
\usepackage{tkz-euclide} % loads  TikZ and tkz-base
\usepackage{listings}
\usepackage{chngcntr}
\usepackage{gvv}
\usepackage[section]{placeins}
%
%\usepackage{setspace}
%\usepackage{gensymb}
%\doublespacing
%\singlespacing

%\usepackage{graphicx}
%\usepackage{amssymb}
%\usepackage{relsize}
%\usepackage[cmex10]{amsmath}
%\usepackage{amsthm}
%\interdisplaylinepenalty=2500
%\savesymbol{iint}
%\usepackage{txfonts}
%\restoresymbol{TXF}{iint}
%\usepackage{wasysym}
%\usepackage{amsthm}
%\usepackage{iithtlc}
%\usepackage{mathrsfs}
%\usepackage{txfonts}
%\usepackage{stfloats}
%\usepackage{bm}
%\usepackage{cite}
%\usepackage{cases}
%\usepackage{subfig}
%\usepackage{xtab}
%\usepackage{longtable}
%\usepackage{multirow}
%\usepackage{algorithm}
%\usepackage{algpseudocode}
%\usepackage{enumitem}
%\usepackage{mathtools}
%\usepackage{tikz}
%\usepackage{circuitikz}
%\usepackage{verbatim}
%\usepackage{tfrupee}
%\usepackage{stmaryrd}
%\usetkzobj{all}
    \usepackage{color}                                            %%
    \usepackage{array}                                            %%
    \usepackage{longtable}                                        %%
    \usepackage{calc}                                             %%
    \usepackage{multirow}                                         %%
    \usepackage{hhline}                                           %%
    \usepackage{ifthen}                                           %%
 %optionally (for landscape tables embedded in another document): %%
    \usepackage{lscape}     
%\usepackage{multicol}
%\usepackage{chngcntr}
%\usepackage{enumerate}

%\usepackage{wasysym}
%\documentclass[conference]{IEEEtran}
%\IEEEoverridecommandlockouts
% The preceding line is only needed to identify funding in the first footnote. If that is unneeded, please comment it out.

\newtheorem{theorem}{Theorem}[section]
\newtheorem{problem}{Problem}
\newtheorem{proposition}{Proposition}[section]
\newtheorem{lemma}{Lemma}[section]
\newtheorem{corollary}[theorem]{Corollary}
\newtheorem{example}{Example}[section]
\newtheorem{definition}[problem]{Definition}
%\newtheorem{thm}{Theorem}[section] 
%\newtheorem{defn}[thm]{Definition}
%\newtheorem{algorithm}{Algorithm}[section]
%\newtheorem{cor}{Corollary}
\newcommand{\BEQA}{\begin{eqnarray}}
\newcommand{\EEQA}{\end{eqnarray}}
\newcommand{\define}{\stackrel{\triangle}{=}}
\theoremstyle{remark}
\newtheorem{rem}{Remark}

%\bibliographystyle{ieeetr}
\begin{document}
%

\bibliographystyle{IEEEtran}


\vspace{3cm}

\title{
%	\logo{
Random Vector Assignment
%	}
}
\author{ Yash Patil
}	
\maketitle

\renewcommand{\thefigure}{\theenumi}
\renewcommand{\thetable}{\theenumi}

Consider a triangle with vertices,\\
\begin{align}
	\vec{A} &= \myvec{4\\-5} ,\\ \vec{B} &= \myvec{-6\\2} ,\\ \vec{C} &= \myvec{-1\\-9}
\end{align}
\section{Vector}
\counterwithin{figure}{section}
\counterwithin{table}{section}
\subsection{Table}
\begin{table}[!htb]
	\input{tables/table1.tex}
	\caption{Equations related to triangle}
	\label{tab58:equations_vector}	
\end{table}
\subsection{Figure}
\begin{figure}[!htb]
	\centering
	\includegraphics[width=\columnwidth]{figs/1.png}
	\caption{Triangle generated using python}
	\label{fig58:Triangle}	
\end{figure}

\section{Median}
\subsection{Table}
\begin{table}[!htb]
	\input{tables/table2.tex}
	\caption{Equations related to median}
	\label{tab58:equations_median}	
\end{table}
\FloatBarrier
\subsection{Figure}
\begin{figure}[!htb]
	\centering
	\includegraphics[width=\columnwidth]{figs/2.png}
	\caption{Triangle with centroid generated using python}
	\label{fig58:Triangle_with_centroid}	
\end{figure}
\begin{figure}[!htb]
	\centering
	\includegraphics[width=\columnwidth]{figs/2_7.png}
	\caption{Proving EAFD is a parallelogram}
	\label{fig58:Triangle_with_altitude}	
\end{figure}
\FloatBarrier

\section{Altitude}
\subsection{Table}
\begin{table}[!htb]
	\input{tables/table3.tex}
	\caption{Equations related to altitude}
	\label{tab58:equations_altitude}	
\end{table}
\subsection{Figure}
\begin{figure}[!htb]
	\centering
	\includegraphics[width=\columnwidth]{figs/3.png}
	\caption{Triangle with altitude generated using python}
	\label{fig58:Triangle_with_altitude}	
\end{figure}

\section{Perpendicular Bisector}
\subsection{Table}
\begin{table}[!htb]
	\input{tables/table4.tex}
	\caption{Equations related to circumcircle}
	\label{tab58:equations_circumcenter}	
\end{table}
\subsection{Figure}
\begin{figure}[!htb]
	\centering
	\includegraphics[width=\columnwidth]{figs/4.png}
	\caption{Triangle with circumcircle generated using python}
	\label{fig58:Triangle_with_circumcircle}	
\end{figure}

\section{Angular Bisector}
\subsection{Table}
\begin{table}[!htb]
	\input{tables/table5.tex}
	\caption{Equations related to incircle}
	\label{tab58:equations_incircle}	
\end{table}
\subsection{Figure}
\begin{figure}[!htb]
	\centering
	\includegraphics[width=\columnwidth]{figs/5.png}
	\caption{Triangle with incircle generated using python}
	\label{fig58:Triangle_with_incircle}	
\end{figure}

\end{document}
